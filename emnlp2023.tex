% This must be in the first 5 lines to tell arXiv to use pdfLaTeX, which is strongly recommended.
\pdfoutput=1
% In particular, the hyperref package requires pdfLaTeX in order to break URLs across lines.

\documentclass[11pt]{article}
\usepackage[review]{EMNLP2023}
\usepackage{times}
\usepackage{latexsym}
\usepackage[T1]{fontenc}
\usepackage[utf8]{inputenc}
\usepackage{microtype}
\usepackage{inconsolata}
\usepackage{tabularray}
\usepackage{tikz}

\definecolor{color1}{HTML}{003F5C}
\definecolor{color2}{HTML}{374C80}
\definecolor{color3}{HTML}{00B7EB}
\definecolor{color4}{HTML}{BC5090}
\definecolor{color5}{HTML}{EF5675}
\definecolor{color6}{HTML}{FF764A}
\definecolor{color7}{HTML}{FFA600}

\UseTblrLibrary{booktabs}

\usetikzlibrary{positioning, arrows.meta, calc, shapes, fit}

% If the title and author information does not fit in the area allocated, uncomment the following
%
%\setlength\titlebox{<dim>}
%
% and set <dim> to something 5cm or larger.

\title{Can an NLP Paper Be this Groundbreaking?}

% Author information can be set in various styles:
% For several authors from the same institution:
% \author{Author 1 \and ... \and Author n \\
%         Address line \\ ... \\ Address line}
% if the names do not fit well on one line use
%         Author 1 \\ {\bf Author 2} \\ ... \\ {\bf Author n} \\
% For authors from different institutions:
% \author{Author 1 \\ Address line \\  ... \\ Address line
%         \And  ... \And
%         Author n \\ Address line \\ ... \\ Address line}
% To start a seperate ``row'' of authors use \AND, as in
% \author{Author 1 \\ Address line \\  ... \\ Address line
%         \AND
%         Author 2 \\ Address line \\ ... \\ Address line \And
%         Author 3 \\ Address line \\ ... \\ Address line}

\author{First Author \\
  Affiliation / Address line 1 \\
  Affiliation / Address line 2 \\
  Affiliation / Address line 3 \\
  \texttt{email@domain} \\\And
  Second Author \\
  Affiliation / Address line 1 \\
  Affiliation / Address line 2 \\
  Affiliation / Address line 3 \\
  \texttt{email@domain} \\}

\begin{document}
\maketitle
\begin{abstract}

  The abstract goes here.

\end{abstract}

\section{Introduction}

%%% Local Variables:
%%% mode: latex
%%% TeX-master: "../emnlp2023"
%%% End:

\section{Background}

\subsection{Motivation}

\subsection{Related Word}

%%% Local Variables:
%%% mode: latex
%%% TeX-master: t
%%% End:

\section{Methodology}

%%% Local Variables:
%%% mode: latex
%%% TeX-master: "../emnlp2023"
%%% End:

\section{Results}

%%% Local Variables:
%%% mode: latex
%%% TeX-master: "../emnlp2023"
%%% End:

\section{Discussion}

%%% Local Variables:
%%% mode: latex
%%% TeX-master: "../emnlp2023"
%%% End:

\section{Future Directions}

%%% Local Variables:
%%% mode: latex
%%% TeX-master: "../emnlp2023"
%%% End:


% Entries for the entire Anthology, followed by custom entries
\bibliographystyle{acl_natbib}
\bibliography{bib/anthology, bib/custom}

\end{document}

%%% Local Variables:
%%% mode: latex
%%% TeX-master: t
%%% End:
